On this system, your adventure takes you to a racous space saloon, swapping tales
with Jan Duet, an infamous smuggler with a heart of gold.

She explains to you that in the early days of hyperspace travel, engines could
instantly transport ships between only certain locations on a six-lightyear continuum.
These options were illustrated using a graph, where the horizontal coordinate
represents starting positions, and the vertical coordinate represents ending positions.

%In Example A illustrated below, a ship at position \(0\) could be teleported
%to any position between \(0\) and \(6\), a ship at position \(0.5\) could only
%be teleported to positions \(4.5\) or \(6\), and a ship at any position between
%\(2\) and \(6\) could only be teleported to position \(6\).

The goal of a hyperspace engine is to be ``ideal'': the collection of possible
destinations from any particular point using exactly one teleportation should be exactly the 
same as the collection of possible destinations that can
be reached from that point using exactly two teleportations.

This means Example A is not ideal. Position \(1\) teleports to
positions \(5\) and \(6\), but from positions \(5\) and \(6\), there are two
problems: a new destination \(3\) can be reached, and the destination \(5\) can
no longer be reached.

However, Example B is ideal. From \(0\), any position can be reached after either
one or two teleportations. From \(2\), positions \(3\) and \(6\) can be reached
after either one or two teleportations. From \(4\), only position \(6\) can be reached
after one or two teleportations. From \(5\), positions \(5\) and \(6\) can be
reached after one or two teleportations. And so on (even for fractional positions!).

Jan suggests that you review your \textbf{Hyperspace Engines} document;
perhaps the illustrations representing ideal engines will reveal a hidden message?

\begin{center}
\begin{tikzpicture}[x=0.2in,y=0.2in] 
\begin{scope}[shift={(0,0)}] 
\node[anchor=south] at (3,6) {Example A: non-ideal};
\fill[white] (0,0) rectangle (6,6);
\draw[step=1,thin,gray] (0,0) grid (6,6);
\draw[line width=2pt] (0,0) -- (0,6) -- (6,6);
\draw[line width=2pt] (0,4) -- (2,6);
\draw[line width=2pt] (3,6) -- (3,3) -- (6,3);
\foreach \x in {0,1,2,3,4,5,6} {
  \node[anchor=north] at (\x,0) {\(\x\)};
}
\foreach \y in {0,1,2,3,4,5,6} {
  \node[anchor=east] at (0,\y) {\(\y\)};
}
\end{scope}
\begin{scope}[shift={(10,0)}] 
\fill[white] (0,0) rectangle (6,6);
\draw[step=1,thin,gray] (0,0) grid (6,6);
\draw[line width=2pt] (0,0) -- (0,6) -- (6,6);
\draw[line width=2pt] (0,4) -- (2,6);
\draw[line width=2pt] (3,6) -- (3,3) -- (6,3);
\foreach \x in {0,2,3,4,5,6} {
  \node[anchor=north] at (\x,0) {\(\x\)};
}
\foreach \y in {0,1,2,3,4} {
  \node[anchor=east] at (0,\y) {\(\y\)};
}
\node[anchor=north,draw,inner sep=2pt,shape=circle] at (1,0) {\(1\)};
\node[anchor=east,draw,inner sep=2pt,shape=circle] at (0,5) {\(5\)};
\node[anchor=east,draw,inner sep=2pt,shape=circle] at (0,6) {\(6\)};
\node[fill=gray,draw,inner sep=2pt,shape=circle] at (1,5) {};
\node[fill=gray,draw,inner sep=2pt,shape=circle] at (1,6) {};
\end{scope}
\begin{scope}[shift={(18,0)}] 
\fill[white] (0,0) rectangle (6,6);
\draw[step=1,thin,gray] (0,0) grid (6,6);
\draw[line width=2pt] (0,0) -- (0,6) -- (6,6);
\draw[line width=2pt] (0,4) -- (2,6);
\draw[line width=2pt] (3,6) -- (3,3) -- (6,3);
\foreach \x in {0,1,2,3,4} {
  \node[anchor=north] at (\x,0) {\(\x\)};
}
\foreach \y in {0,1,2,4} {
  \node[anchor=east] at (0,\y) {\(\y\)};
}
\node[anchor=north,draw,inner sep=2pt,shape=circle] at (5,0) {\(5\)};
\node[anchor=north,draw,inner sep=2pt,shape=circle] at (6,0) {\(6\)};
\node[anchor=east,draw,cross out] at (0,5) {\(5\)};
\node[anchor=east,draw,inner sep=2pt,shape=circle] at (0,6) {\(6\)};
\node[anchor=east,draw,inner sep=2pt,shape=circle] at (0,3) {\(3\)};
\node[anchor=east,draw,inner sep=4pt,cross out] at (-0.2,3) {};
\node[fill=gray,draw,inner sep=2pt,shape=circle] at (5,6) {};
\node[fill=gray,draw,inner sep=2pt,shape=circle] at (6,6) {};
\node[fill=gray,draw,inner sep=2pt,shape=circle] at (5,3) {};
\node[fill=gray,draw,inner sep=2pt,shape=circle] at (6,3) {};
\end{scope}

\begin{scope}[shift={(0,-10)}] 
\node[anchor=south] at (3,6) {Example B: ideal};
\fill[white] (0,0) rectangle (6,6);
\draw[step=1,thin,gray] (0,0) grid (6,6);
\draw[line width=2pt] (0,0) -- (0,6) -- (6,6) -- (5,5);
\draw[line width=2pt] (0,3) -- (3,3);
\foreach \x in {0,1,2,3,4,5,6} {
  \node[anchor=north] at (\x,0) {\(\x\)};
}
\foreach \y in {0,1,2,3,4,5,6} {
  \node[anchor=east] at (0,\y) {\(\y\)};
}
\end{scope}
\begin{scope}[shift={(10,-10)}] 
\fill[white] (0,0) rectangle (6,6);
\draw[step=1,thin,gray] (0,0) grid (6,6);
\draw[line width=2pt] (0,0) -- (0,6) -- (6,6) -- (5,5);
\draw[line width=2pt] (0,3) -- (3,3);
\foreach \x in {0,1,3,4,5,6} {
  \node[anchor=north] at (\x,0) {\(\x\)};
}
\foreach \y in {0,1,2,4,5} {
  \node[anchor=east] at (0,\y) {\(\y\)};
}
\node[anchor=north,draw,inner sep=2pt,shape=circle] at (2,0) {\(2\)};
\node[anchor=east,draw,inner sep=2pt,shape=circle] at (0,3) {\(3\)};
\node[anchor=east,draw,inner sep=2pt,shape=circle] at (0,6) {\(6\)};
\node[fill=gray,draw,inner sep=2pt,shape=circle] at (2,3) {};
\node[fill=gray,draw,inner sep=2pt,shape=circle] at (2,6) {};
\end{scope}
\begin{scope}[shift={(18,-10)}] 
\fill[white] (0,0) rectangle (6,6);
\draw[step=1,thin,gray] (0,0) grid (6,6);
\draw[line width=2pt] (0,0) -- (0,6) -- (6,6) -- (5,5);
\draw[line width=2pt] (0,3) -- (3,3);
\foreach \x in {0,1,2,4,5} {
  \node[anchor=north] at (\x,0) {\(\x\)};
}
\foreach \y in {0,1,2,4,5} {
  \node[anchor=east] at (0,\y) {\(\y\)};
}
\node[anchor=north,draw,inner sep=2pt,shape=circle] at (3,0) {\(3\)};
\node[anchor=north,draw,inner sep=2pt,shape=circle] at (6,0) {\(6\)};
\node[anchor=east,draw,inner sep=2pt,shape=circle] at (0,3) {\(3\)};
\node[anchor=east,draw,inner sep=2pt,shape=circle] at (0,6) {\(6\)};
\node[fill=gray,draw,inner sep=2pt,shape=circle] at (3,3) {};
\node[fill=gray,draw,inner sep=2pt,shape=circle] at (3,6) {};
\node[fill=gray,draw,inner sep=2pt,shape=circle] at (6,6) {};
\end{scope}
\end{tikzpicture}
\end{center}

 
