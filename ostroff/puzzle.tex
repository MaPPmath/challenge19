%!TEX root = ../booklet.tex
% ^ leave for LaTeXTools build functionality

\newcommand*\circled[1]{\tikz[baseline=(char.base)]{
            \node[shape=circle,draw,inner sep=2pt] (char) {#1};}}
\newcommand*\markedout[1]{\tikz[baseline=(char.base)]{
            \node[shape=rectangle,draw,inner sep=2pt,fill=black] (char) {#1};}}

\begin{puzzle}
There are \(6\) sides on a normal die. So, as you probably know,
the probability of rolling a four is \(1\) out of \(6\).

\begin{center}\Large
  \epsdice{1}
  \epsdice{2}
  \epsdice{3}
  \circled{\epsdice{4}}
  \epsdice{5}
  \epsdice{6}
\end{center}

\vfill

There are \(6\cdot 6=36\) different combinations of
faces for two six-sided
dice. As shown below, the probability that a roll contains
either a three, or a five, but not both, is \(18\) out of \(36\).
Since \(\frac{18}{36}=\frac{1}{2}\), this
is the same as \(1\) out of \(2\).

\vfill

\begin{center}\Large
  \epsdice{1}\epsdice{1}
  \epsdice{1}\epsdice{2}
  \circled{\epsdice{1}\epsdice{3}}
  \epsdice{1}\epsdice{4}
  \circled{\epsdice{1}\epsdice{5}}
  \epsdice{1}\epsdice{6}
\end{center}
\begin{center}\Large
  \epsdice{2}\epsdice{1}
  \epsdice{2}\epsdice{2}
  \circled{\epsdice{2}\epsdice{3}}
  \epsdice{2}\epsdice{4}
  \circled{\epsdice{2}\epsdice{5}}
  \epsdice{2}\epsdice{6}
\end{center}
\begin{center}\Large
  \circled{\epsdice{3}\epsdice{1}}
  \circled{\epsdice{3}\epsdice{2}}
  \circled{\epsdice{3}\epsdice{3}}
  \circled{\epsdice{3}\epsdice{4}}
  \epsdice{3}\epsdice{5}
  \circled{\epsdice{3}\epsdice{6}}
\end{center}
\begin{center}\Large
  \epsdice{4}\epsdice{1}
  \epsdice{4}\epsdice{2}
  \circled{\epsdice{4}\epsdice{3}}
  \epsdice{4}\epsdice{4}
  \circled{\epsdice{4}\epsdice{5}}
  \epsdice{4}\epsdice{6}
\end{center}
\begin{center}\Large
  \circled{\epsdice{5}\epsdice{1}}
  \circled{\epsdice{5}\epsdice{2}}
  \epsdice{5}\epsdice{3}
  \circled{\epsdice{5}\epsdice{4}}
  \circled{\epsdice{5}\epsdice{5}}
  \circled{\epsdice{5}\epsdice{6}}
\end{center}
\begin{center}\Large
  \epsdice{6}\epsdice{1}
  \epsdice{6}\epsdice{2}
  \circled{\epsdice{6}\epsdice{3}}
  \epsdice{6}\epsdice{4}
  \circled{\epsdice{6}\epsdice{5}}
  \epsdice{6}\epsdice{6}
\end{center}

\vfill

\newpage

This can get complicated if rerolls are sometimes allowed.
This can be shown by marking out the cases which would have to be rerolled.
For example, what's the probability that the sum of two dice is 6 or 7,
if any sums of 4 are rerolled? To solve this, we can mark
out the pairs which add to 4, and circle the pairs
adding to 6 or 7.

\vfill

\begin{center}\Huge
  \epsdice{1}\epsdice{1}
  \epsdice{1}\epsdice{2}
  \markedout{\epsdice{1}\epsdice{3}}
  \epsdice{1}\epsdice{4}
  \circled{\epsdice{1}\epsdice{5}}
  \circled{\epsdice{1}\epsdice{6}}
\end{center}
\begin{center}\Huge
  \epsdice{2}\epsdice{1}
  \markedout{\epsdice{2}\epsdice{2}}
  \epsdice{2}\epsdice{3}
  \circled{\epsdice{2}\epsdice{4}}
  \circled{\epsdice{2}\epsdice{5}}
  \epsdice{2}\epsdice{6}
\end{center}
\begin{center}\Huge
  \markedout{\epsdice{3}\epsdice{1}}
  \epsdice{3}\epsdice{2}
  \circled{\epsdice{3}\epsdice{3}}
  \circled{\epsdice{3}\epsdice{4}}
  \epsdice{3}\epsdice{5}
  \epsdice{3}\epsdice{6}
\end{center}
\begin{center}\Huge
  \epsdice{4}\epsdice{1}
  \circled{\epsdice{4}\epsdice{2}}
  \circled{\epsdice{4}\epsdice{3}}
  \epsdice{4}\epsdice{4}
  \epsdice{4}\epsdice{5}
  \epsdice{4}\epsdice{6}
\end{center}
\begin{center}\Huge
  \circled{\epsdice{5}\epsdice{1}}
  \circled{\epsdice{5}\epsdice{2}}
  \epsdice{5}\epsdice{3}
  \epsdice{5}\epsdice{4}
  \epsdice{5}\epsdice{5}
  \epsdice{5}\epsdice{6}
\end{center}
\begin{center}\Huge
  \circled{\epsdice{6}\epsdice{1}}
  \epsdice{6}\epsdice{2}
  \epsdice{6}\epsdice{3}
  \epsdice{6}\epsdice{4}
  \epsdice{6}\epsdice{5}
  \epsdice{6}\epsdice{6}
\end{center}

\vfill

This means that the probability is \(11\) out
of the remaining \(33\), so \(1\) out of \(3\).

\vfill
\newpage

So, can you figure out the probability of each of the following results?

\begin{center}\begin{tabular}{m{2.8 in}|l|l}
Action & Outcome & Probability \\ \hline \hline
Tina rolled two dice. & The sum was 5. & \(1\) out of... \hspace{3em} \\  \hline
Tina rolled two dice. & The sum was a multiple of 3. & \(1\) out of... \hspace{3em} \\  \hline
Tina rolled two dice, rerolling both dice if 6 appears. & The sum was 6. & \(1\) out of... \hspace{3em} \\  \hline \hline
Tina rolled two dice. & The product was odd. & \(1\) out of... \hspace{3em} \\  \hline
Tina rolled two dice, rerolling snake eyes (pairs of \(1\)s). & The product was a square number. & \(1\) out of... \hspace{3em} \\  \hline
Tina rolled two dice. & The product was 4. & \(1\) out of... \hspace{3em} \\  \hline
Tina rolled two dice. & The product was 12. & \(1\) out of... \hspace{3em} \\  \hline
Tina rolled two dice, rerolling both dice if the product contained the letter ``n'' (e.g. ``o\textbf{n}e'', ``fourtee\textbf{n}''). & The sum and product were equal. & \(1\) out of... \hspace{3em} \\  \hline
Tina rolled two dice, rerolling if the two faces were on opposite sides of the die. & The two faces were the same. & \(1\) out of... \hspace{3em} \\  \hline
Tina rolled two dice. & The difference was \(\pm5\). & \(1\) out of... \hspace{3em} \\  \hline
Tina rolled two dice, rerolling both dice whenever the product was a multiple of 5. & The sum was 12. & \(1\) out of... \hspace{3em}  \\
\end{tabular}\end{center}

\vfill

The above results spell a message, assuming \(A=1\), \(B=2\),
and so on. \textbf{Report
this message to Game HQ for \(100\) Victory Points!}

\vfill

\begin{center}
    \letterBox{}
    \letterBox{}
    \letterBox{}
\end{center}
\begin{center}
    \letterBox{}
    \letterBox{}
    \letterBox{}
    \letterBox{}
    \letterBox{}
    \letterBox{}
    \letterBox{}
    \letterBox{}
\end{center}

\vfill




\end{puzzle}

\begin{extraPuzzle}
Now Tina's invited a couple friends over to play a board game, but they
can't decide who gets to go first. Tina suggests that they should choose
three dice from the following strange collection:

\begin{itemize}
\item Die A has sides labeled [1, 27, 38, 41, 43, 46]
\item Die B has sides labeled [2, 10, 45, 56, 57, 59]
\item Die C has sides labeled [3, 12, 24, 34, 36, 50]
\item Die D has sides labeled [4, 28, 30, 44, 54, 58]
\item Die E has sides labeled [5, 15, 20, 25, 33, 39]
\item Die F has sides labeled [6, 18, 21, 26, 32, 55]
\item Die G has sides labeled [7, 19, 31, 49, 51, 53]
\item Die H has sides labeled [8, 14, 17, 22, 40, 60]
\item Die I has sides labeled [9, 11, 16, 35, 42, 47]
\item Die J has sides labeled [13, 23, 29, 37, 48, 52]
\end{itemize}

Each player will throw one of the three chosen dice, and the player rolling
the highest goes first. For three dice, there are \(6^3=216\) possible
outcomes. If the combination is ``Go-First fair'', then each die wins
exactly \(\frac{216}{3}=72\) of those outcomes. However, most of the
above dice combinations are unfair: out of dice ABC, B wins \(134\) of the
outcomes, A only wins \(51\) outcomes, and C wins just \(31\).

Submit a combination of the three above dice which you think is Go-First fairest.
\textbf{The team(s) submitting the fairest combination of dice
will receive \(50\) Victory Points.} By the way, a perfect combination of
Go First Dice was first
invented in 2010 by puzzle designer Eric Harshbarger and mathematician
Dr. Robert Ford. Tina is certain she has that exact set lying around somewhere...

\begin{center}
    \letterBox{}
    \letterBox{}
    \letterBox{}
\end{center}

\end{extraPuzzle}



\begin{puzzleSolutions}
FOR STAFF USE ONLY

Main puzzle solution: ICE DELIVERY

To grade the Extra, compare the submission with the below list.
We wrote a computer program to compute the ``max advantage'' of each combination
of dice, where the advantage of each die is the number of outcomes it wins
out of all \(6^3\) possible outcomes.
Award points to the team(s) which choose a combination closest to
the completely fair max advantage: \(6^3/3=72\).

\begin{multicols}{2}\small
\begin{itemize}
\item{} [A, B, C] has max advantage 134
\item{} [A, B, D] has max advantage 117
\item{} [A, B, E] has max advantage 139
\item{} [A, B, F] has max advantage 134
\item{} [A, B, G] has max advantage 124
\item{} [A, B, H] has max advantage 116
\item{} [A, B, I] has max advantage 134
\item{} [A, B, J] has max advantage 128
\item{} [A, C, D] has max advantage 110
\item{} [A, C, E] has max advantage 127
\item{} [A, C, F] has max advantage 112
\item{} [A, C, G] has max advantage 111
\item{} [A, C, H] has max advantage 107
\item{} [A, C, I] has max advantage 99
\item{} [A, C, J] has max advantage 86
\item{} [A, D, E] has max advantage 118
\item{} [A, D, F] has max advantage 107
\item{} [A, D, G] has max advantage 95
\item{} [A, D, H] has max advantage 101
\item{} [A, D, I] has max advantage 109
\item{} [A, D, J] has max advantage 102
\item{} [A, E, F] has max advantage 131
\item{} [A, E, G] has max advantage 119
\item{} [A, E, H] has max advantage 126
\item{} [A, E, I] has max advantage 116
\item{} [A, E, J] has max advantage 100
\item{} [A, F, G] has max advantage 101
\item{} [A, F, H] has max advantage 111
\item{} [A, F, I] has max advantage 102
\item{} [A, F, J] has max advantage 88
\item{} [A, G, H] has max advantage 101
\item{} [A, G, I] has max advantage 117
\item{} [A, G, J] has max advantage 103
\item{} [A, H, I] has max advantage 98
\item{} [A, H, J] has max advantage 84
\item{} [A, I, J] has max advantage 91
\item{} [B, C, D] has max advantage 117
\item{} [B, C, E] has max advantage 139
\item{} [B, C, F] has max advantage 134
\item{} [B, C, G] has max advantage 124
\item{} [B, C, H] has max advantage 116
\item{} [B, C, I] has max advantage 134
\item{} [B, C, J] has max advantage 128
\item{} [B, D, E] has max advantage 121
\item{} [B, D, F] has max advantage 117
\item{} [B, D, G] has max advantage 109
\item{} [B, D, H] has max advantage 101
\item{} [B, D, I] has max advantage 117
\item{} [B, D, J] has max advantage 112
\item{} [B, E, F] has max advantage 139
\item{} [B, E, G] has max advantage 127
\item{} [B, E, H] has max advantage 121
\item{} [B, E, I] has max advantage 139
\item{} [B, E, J] has max advantage 132
\item{} [B, F, G] has max advantage 124
\item{} [B, F, H] has max advantage 116
\item{} [B, F, I] has max advantage 134
\item{} [B, F, J] has max advantage 128
\item{} [B, G, H] has max advantage 106
\item{} [B, G, I] has max advantage 124
\item{} [B, G, J] has max advantage 120
\item{} [B, H, I] has max advantage 116
\item{} [B, H, J] has max advantage 110
\item{} [B, I, J] has max advantage 128
\item{} [C, D, E] has max advantage 126
\item{} [C, D, F] has max advantage 115
\item{} [C, D, G] has max advantage 99
\item{} [C, D, H] has max advantage 109
\item{} [C, D, I] has max advantage 115
\item{} [C, D, J] has max advantage 107
\item{} [C, E, F] has max advantage 90
\item{} [C, E, G] has max advantage 119
\item{} [C, E, H] has max advantage 83
\item{} [C, E, I] has max advantage 86
\item{} [C, E, J] has max advantage 111
\item{} [C, F, G] has max advantage 102
\item{} [C, F, H] has max advantage 78
\item{} [C, F, I] has max advantage 76
\item{} [C, F, J] has max advantage 100
\item{} [C, G, H] has max advantage 103
\item{} [C, G, I] has max advantage 117
\item{} [C, G, J] has max advantage 102
\item{} \textbf{[C, H, I] has max advantage 72}
\item{} [C, H, J] has max advantage 97
\item{} [C, I, J] has max advantage 105
\item{} [D, E, F] has max advantage 128
\item{} [D, E, G] has max advantage 106
\item{} [D, E, H] has max advantage 122
\item{} [D, E, I] has max advantage 126
\item{} [D, E, J] has max advantage 116
\item{} [D, F, G] has max advantage 97
\item{} [D, F, H] has max advantage 112
\item{} [D, F, I] has max advantage 115
\item{} [D, F, J] has max advantage 106
\item{} [D, G, H] has max advantage 91
\item{} [D, G, I] has max advantage 99
\item{} [D, G, J] has max advantage 94
\item{} [D, H, I] has max advantage 109
\item{} [D, H, J] has max advantage 100
\item{} [D, I, J] has max advantage 107
\item{} [E, F, G] has max advantage 111
\item{} [E, F, H] has max advantage 79
\item{} [E, F, I] has max advantage 89
\item{} [E, F, J] has max advantage 111
\item{} [E, G, H] has max advantage 112
\item{} [E, G, I] has max advantage 126
\item{} [E, G, J] has max advantage 110
\item{} [E, H, I] has max advantage 86
\item{} [E, H, J] has max advantage 109
\item{} [E, I, J] has max advantage 115
\item{} [F, G, H] has max advantage 97
\item{} [F, G, I] has max advantage 108
\item{} [F, G, J] has max advantage 94
\item{} [F, H, I] has max advantage 74
\item{} [F, H, J] has max advantage 99
\item{} [F, I, J] has max advantage 103
\item{} [G, H, I] has max advantage 111
\item{} [G, H, J] has max advantage 95
\item{} [G, I, J] has max advantage 108
\item{} [H, I, J] has max advantage 102
\end{itemize}
\end{multicols}
\end{puzzleSolutions}

