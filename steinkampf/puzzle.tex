%!TEX root = ../booklet.tex
% ^ leave for LaTeXTools build functionality

\begin{puzzle}
Welcome to Simple City! The city planners here have
designed their neighborhoods in \(4\times 4\) grids. Each row and each column
is composed exactly of four towers which are 1, 2, 3, and 4 stories tall
respectively. Also, the planners like to know how visible these towers
can be, so they record \textbf{boundary conditions} which count the number
of visible towers from each position on the outside edge of the neighborhood.
For example, a 1 indicates that the tallest tower is on the edge and blocking
all the other towers, and a 4 indicates that all the towers in that row or
column are visible from that position.

\vfill

\begin{center}
  \fbox{\begin{minipage}{.7\linewidth}
    \textbf{Example} Here's a completed neighborhood with its boundary
    conditions. For instance, the 3 in the upper left corner represents a person
    looking south into the \(4\times 4\) grid; she can only see the towers
    with heights \(1,3,4\) since the \(2\) is blocked by the \(3\).
\begin{center}\Large
\begin{tabular}{c||c|c|c|c||c}
  & 3\(\downarrow\)         & 2\(\downarrow\)         & 2\(\downarrow\)         & 1\(\downarrow\)         &   \\\hline\hline
4\(\rightarrow\) & 1         & 2         & 3         & 4         & \(\leftarrow\)1 \\\hline
2\(\rightarrow\) & 3         & 4         & 2         & 1         & \(\leftarrow\)3 \\\hline
2\(\rightarrow\) & 2         & 1         & 4         & 3         & \(\leftarrow\)2 \\\hline
1\(\rightarrow\) & 4         & 3         & 1         & 2         & \(\leftarrow\)3 \\\hline\hline
  & 1\(\uparrow\)         & 2\(\uparrow\)         & 2\(\uparrow\)         & 3\(\uparrow\)         &
\end{tabular}
\end{center}
  \end{minipage}}
\end{center}

\vfill

Your goal is to figure out the heights of the towers
in each neighborhood on the following page using only the boundary
conditions.
Some of the towers have been marked with a symbol,
and some of the boundary conditions are missing.

\vfill

\newpage

\begin{multicols}{2}
\begin{center}
\begin{tabular}{c||c|c|c|c||c}
  & 3\(\downarrow\)         & 2\(\downarrow\)         & 2\(\downarrow\)         & 1\(\downarrow\)         &   \\\hline\hline
4\(\rightarrow\) & \(\star\) &           &           &           & \(\leftarrow\)1 \\\hline
2\(\rightarrow\) &           &           & \(\#\)    &           & \(\leftarrow\)2 \\\hline
3\(\rightarrow\) &           &           &           &           & \(\leftarrow\)2 \\\hline
1\(\rightarrow\) &           &           &           & \(=\)     & \(\leftarrow\)2 \\\hline\hline
  & 1\(\uparrow\)         & 3\(\uparrow\)         & 2\(\uparrow\)         & 2\(\uparrow\)         &
\end{tabular}

\begin{tabular}{c||c|c|c|c||c}
  & 2\(\downarrow\)         &           & 2\(\downarrow\)         & ~~        &   \\\hline\hline
3\(\rightarrow\) &           &           &           &           & \(\leftarrow\)1 \\\hline
3\(\rightarrow\) &           & \(+\)     & \(\sim\)  &           &   \\\hline
1\(\rightarrow\) &           &           &           &           & \(\leftarrow\)3 \\\hline
  & \(\%\)    &           &           &           &   \\\hline\hline
  & 2\(\uparrow\)         & 1\(\uparrow\)         & 3\(\uparrow\)         &           &
\end{tabular}
\end{center}

\begin{center}
\begin{tabular}{c||c|c|c|c||c}
  & ~~        & 2\(\downarrow\)         & 3\(\downarrow\)         &           &   \\\hline\hline
  &           &           &           &           & \(\leftarrow\)3 \\\hline
3\(\rightarrow\) &           &           &           & \(\&\)    &   \\\hline
  &           &           & \(@\)     &           & \(\leftarrow\)3 \\\hline
  &           & \(\$\)    &           &           &   \\\hline\hline
  &           &           &           &           &
\end{tabular}

\begin{tabular}{c||c|c|c|c||c}
  &           &           & 3\(\downarrow\)         &           &   \\\hline\hline
  & \(?\)     &           &           &           &   \\\hline
3\(\rightarrow\) &           &           &           &           &   \\\hline
2\(\rightarrow\) &           & \(\sqrt{}\) &         & \(\partial\) & \(\leftarrow\)2 \\\hline
2\(\rightarrow\) &           &           &           &           &   \\\hline\hline
  &           & 1\(\uparrow\)         &           & 3\(\uparrow\)         &
\end{tabular}
\end{center}
\end{multicols}

\vfill


For each of the above symbols, mark the height of its tower in the chart
below to reveal a message.

\begin{center}
\begin{tabular}{c||c|c|c|c|c||c|c|c|c|c|c|c}
& \(\star\) & \(?\) & \(@\) & \(\$\) & \(\%\) & \(\#\) & \(\sim\)
& \(\partial\) & \(\&\) & \(=\) & \(+\) & \(\sqrt{}\) \\\hline\hline
4 & A & A & I & M & N & K & I & G & H & T & G & R \\
3 & F & D & M & I & H & I & E & R & U & O & N & Y \\
2 & S & C & U & G & E & C & N & T & T & N & U & S \\
1 & L & L & A & E & R & W & N & F & I & I & O & T
\end{tabular}
\end{center}

\begin{center}
    \letterBox{}
    \letterBox{}
    \letterBox{}
    \letterBox{}
    \letterBox{}
\end{center}
\begin{center}
    \letterBox{}
    \letterBox{}
    \letterBox{}
    \letterBox{}
    \letterBox{}
    \letterBox{}
    \letterBox{}
\end{center}

\vfill

\noindent
\textbf{Report the decoded message to Game HQ for \(100\) Victory Points!}

\vfill

\end{puzzle}

\begin{extraPuzzle}
The citizens of Simple City have begun work on a new burough called
Sudoku City. This neighborhood contains plots for up to \(9\times 9\) towers,
and ordinances require that each column, row, and \(3\times 3\) square
contains at most one tower with height \(1\) through \(9\), just like a
Sudoku puzzle. The numbers below
represent the desired boundary conditions for this neighborhood, just like
in the main puzzle.

Your task is to build as many towers as possible in the \(9\times 9\) grid
by labeling each plot with a height for its tower.
\textit{You do not have to fill
out all the plots or fulfill all boundary conditions},
but you should \textit{never} violate the Sudoku rules by having
more than one tower with the same height in a row, column, or \(3\times 3\)
square.

Submit the filled out grid to Game HQ before the
end of the game. Grids violating any Sudoku rules will be disqualified.
The judges will award \(1\) puzzle point for each of the eighty-one
plots used and
\(100\) puzzle points for each of the thirty-six boundary conditions
satisfied along the border.
\textbf{The team(s) earning the highest amount of puzzle points
will earn \(50\) Victory Points.}


\begin{center}\Large
\begin{tabular}{c||c|c|c||c|c|c||c|c|c||c}
  & 3\(\downarrow\) & 4\(\downarrow\) & 1\(\downarrow\) & 2\(\downarrow\) & 2\(\downarrow\) & 5\(\downarrow\) & 3\(\downarrow\) & 3\(\downarrow\) & 3\(\downarrow\) &   \\\hline\hline
3\(\rightarrow\) &   &   &   &   &   &   &   &   &   & \(\leftarrow\)3 \\\hline
3\(\rightarrow\) &   &   &   &   &   &   &   &   &   & \(\leftarrow\)4 \\\hline
3\(\rightarrow\) &   &   &   &   &   &   &   &   &   & \(\leftarrow\)2 \\\hline\hline
1\(\rightarrow\) &   &   &   &   &   &   &   &   &   & \(\leftarrow\)4 \\\hline
3\(\rightarrow\) &   &   &   &   &   &   &   &   &   & \(\leftarrow\)2 \\\hline
2\(\rightarrow\) &   &   &   &   &   &   &   &   &   & \(\leftarrow\)2 \\\hline\hline
3\(\rightarrow\) &   &   &   &   &   &   &   &   &   & \(\leftarrow\)1 \\\hline
3\(\rightarrow\) &   &   &   &   &   &   &   &   &   & \(\leftarrow\)3 \\\hline
2\(\rightarrow\) &   &   &   &   &   &   &   &   &   & \(\leftarrow\)3 \\\hline\hline
  & 5\(\uparrow\) & 1\(\uparrow\) & 3\(\uparrow\) & 3\(\uparrow\) & 4\(\uparrow\) & 2\(\uparrow\) & 2\(\uparrow\) & 2\(\uparrow\) & 2\(\uparrow\) &
\end{tabular}
\end{center}
\end{extraPuzzle}



\begin{puzzleSolutions}
FOR STAFF USE ONLY

Main puzzle solution: LAUGH WITHOUT

To grade the Extra,
check that the submission satisfies the Sudoku conditions: each row,
column, and \(3\times3\) square contains at most one of the numbers
\(1,\dots,9\) (blanks are fine). Disqualify any submissions which fail that check.
Otherwise, add \(100\) for each boundary condition satisfied, and \(1\)
for each plot used; the submission with the largest sum wins.

Here's a perfect solution (from which this puzzle was generated):

\begin{center}\Large
\begin{tabular}{c||c|c|c||c|c|c||c|c|c||c}
  & 3\(\downarrow\) & 4\(\downarrow\) & 1\(\downarrow\) & 2\(\downarrow\) & 2\(\downarrow\) & 5\(\downarrow\) & 3\(\downarrow\) & 3\(\downarrow\) & 3\(\downarrow\) &   \\\hline\hline
3\(\rightarrow\) & 1 & 3 & 9 & 6 & 8 & 5 & 4 & 2 & 7 & \(\leftarrow\)3 \\\hline
3\(\rightarrow\) & 5 & 7 & 4 & 9 & 3 & 2 & 8 & 6 & 1 & \(\leftarrow\)4 \\\hline
3\(\rightarrow\) & 2 & 8 & 6 & 1 & 7 & 4 & 5 & 9 & 3 & \(\leftarrow\)2 \\\hline\hline
1\(\rightarrow\) & 9 & 6 & 7 & 8 & 1 & 3 & 2 & 5 & 4 & \(\leftarrow\)4 \\\hline
3\(\rightarrow\) & 4 & 2 & 3 & 5 & 9 & 6 & 7 & 1 & 8 & \(\leftarrow\)2 \\\hline
2\(\rightarrow\) & 8 & 5 & 1 & 4 & 2 & 7 & 9 & 3 & 6 & \(\leftarrow\)2 \\\hline\hline
3\(\rightarrow\) & 7 & 1 & 5 & 2 & 6 & 8 & 3 & 4 & 9 & \(\leftarrow\)1 \\\hline
3\(\rightarrow\) & 6 & 4 & 8 & 3 & 5 & 9 & 1 & 7 & 2 & \(\leftarrow\)3 \\\hline
2\(\rightarrow\) & 3 & 9 & 2 & 7 & 4 & 1 & 6 & 8 & 5 & \(\leftarrow\)3 \\\hline\hline
  & 5\(\uparrow\) & 1\(\uparrow\) & 3\(\uparrow\) & 3\(\uparrow\) & 4\(\uparrow\) & 2\(\uparrow\) & 2\(\uparrow\) & 2\(\uparrow\) & 2\(\uparrow\) &
\end{tabular}
\end{center}
\end{puzzleSolutions}
